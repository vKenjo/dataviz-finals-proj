\documentclass[sigconf]{acmart}

\usepackage{graphicx}
\usepackage{booktabs}
\usepackage{hyperref}
\usepackage{subcaption}

%%
%% Rights management information.
\setcopyright{none}

%%
%% Submission ID.
\acmSubmissionID{123-A45-678}

%%
%% The majority of ACM publications use numbered citations and
%% references.
\citestyle{acmauthoryear}

%%
%% end of the preamble, start of the body of the document source.
\begin{document}

%%
%% The "title" command has an optional parameter,
%% allowing the author to define a "short title" to be used in page headers.
\title{Visualizing Global Poverty Trends: An Analysis of SDG 1 Progress with Focus on the Philippines and ASEAN}

%%
%% The "author" command and its associated commands are used to define
%% the authors and their affiliations.
\author{CS ELEC 3C Data Visualization Project}
\affiliation{%
  \institution{December 10, 2025}
}

%%
%% The abstract is a short summary of the work to be presented in the
%% article.
\begin{abstract}
This project analyzes global poverty trends from 1960-2024 using World Development Indicators data, addressing UN Sustainable Development Goal 1: No Poverty. Through comprehensive visualization and statistical analysis of 217 countries, we examine extreme poverty reduction trajectories, with specific focus on the Philippines and ASEAN region. Our findings reveal dramatic global progress—poverty reduced from 17.22\% (1981) to 4.23\% (2024)—yet significant regional disparities persist. Key insights include: infrastructure access (electricity) shows the strongest correlation with poverty reduction (r = -0.847), surpassing GDP per capita (r = -0.397); ASEAN countries achieved a median poverty rate of 5.4\%, with success stories like Malaysia and Thailand reaching 0\%; and the Philippines, at 5.3\% poverty, performs slightly below ASEAN median but above the global median of 2.7\%. Interactive visualizations including choropleth maps, time-series analyses, correlation matrices, and radar charts reveal that successful poverty reduction requires simultaneous progress in infrastructure, health, and economic development. These evidence-based findings provide actionable policy recommendations for accelerating progress toward the 2030 SDG deadline.
\end{abstract}

%%
%% Keywords. The author(s) should pick words that accurately describe
%% the work being presented. Separate the keywords with commas.
\keywords{data visualization, poverty reduction, sustainable development goals, World Development Indicators, ASEAN, Philippines, choropleth maps, correlation analysis}

%%
%% This command processes the author and affiliation and title
%% information and builds the first part of the formatted document.
\maketitle

\section{Project Background}

\subsection{The Global Poverty Challenge}

Extreme poverty remains one of humanity's most pressing challenges. Despite remarkable economic growth and technological advancement, approximately 700 million people—nearly 9\% of the global population—still live on less than \$2.15 per day as of 2023 \cite{worldbank2023}.

\subsection{UN Sustainable Development Goal 1: No Poverty}

The United Nations' Sustainable Development Goal 1 aims to \textbf{``end poverty in all its forms everywhere by 2030''} \cite{un2015sdg}. This ambitious target encompasses:

\begin{itemize}
\item Eradicating extreme poverty (living on <\$2.15/day)
\item Reducing poverty by half according to national definitions
\item Implementing social protection systems
\item Ensuring equal rights to economic resources and basic services
\end{itemize}

\subsection{Historical Context}

The past three decades have witnessed unprecedented poverty reduction:
\begin{itemize}
\item \textbf{1990:} $\sim$36\% of the global population lived in extreme poverty
\item \textbf{2023:} $\sim$9\% live in extreme poverty
\item \textbf{Result:} Over 1.2 billion people lifted out of poverty
\end{itemize}

However, progress has been uneven:
\begin{itemize}
\item \textbf{East Asia:} Reduced from 66\% to <1\% (driven by China)
\item \textbf{Sub-Saharan Africa:} Remains above 40\% in many countries
\end{itemize}

\subsection{Project Significance}

This project contributes by:
\begin{enumerate}
\item Providing comprehensive visual analysis of 33 years of global poverty data
\item Identifying correlations between poverty and development indicators
\item Comparing successful and challenging cases
\item Generating evidence-based policy recommendations
\end{enumerate}

\section{Statement of the Problem}

\subsection{Primary Research Question}

\textbf{``How has global poverty evolved from 1960-2024, and what is the Philippines' position relative to ASEAN countries and global trends in terms of poverty and development indicators?''}

\subsection{Specific Research Questions}

\paragraph{Global Level Questions}
\begin{itemize}
\item \textbf{RQ1}: How has extreme poverty evolved globally from 1981-2024 across 217 countries?
\item \textbf{RQ2}: What are the geographic patterns and regional disparities in poverty rates?
\item \textbf{RQ3}: How do development indicators (GDP per capita, life expectancy, education, infrastructure) correlate with poverty globally?
\end{itemize}

\paragraph{ASEAN Regional Questions}
\begin{itemize}
\item \textbf{RQ4}: How does the Philippines compare to other ASEAN countries in poverty rates?
\item \textbf{RQ5}: Which ASEAN countries have been most/least successful in reducing poverty?
\item \textbf{RQ6}: Where does ASEAN stand relative to global poverty levels?
\end{itemize}

\paragraph{Philippines-Specific Questions}
\begin{itemize}
\item \textbf{RQ7}: What is the Philippines' current poverty status and how has it evolved over time?
\item \textbf{RQ8}: How does the Philippines perform across multiple development dimensions compared to the ASEAN average?
\item \textbf{RQ9}: What development indicators show the strongest correlation with poverty reduction?
\end{itemize}

\subsection{Scope and Limitations}

\paragraph{Geographic Coverage}
\begin{itemize}
\item \textbf{Global Analysis}: 217 countries with World Bank data (169 with poverty data, 48 without)
\item \textbf{ASEAN Focus} (11 member states): Brunei, Cambodia, Indonesia, Lao PDR, Malaysia, Myanmar, Philippines, Singapore, Thailand, Timor-Leste, Vietnam
\item \textbf{Philippines Deep-Dive}: Complete available time series with multi-indicator comparison
\end{itemize}

\paragraph{Temporal Coverage}
Overall period covers 1960-2024 (65 years), with poverty data primarily from 1981-2024. Most countries have data from 1990s onward, with latest year varying by country (2018-2024).

\paragraph{Limitations}
\begin{enumerate}
\item \textbf{Data Availability}: Poverty data is sparse before 1990; only 169 of 217 countries have measurements
\item \textbf{ASEAN Data Gaps}: Only 7 of 11 ASEAN countries have recent poverty data
\item \textbf{Temporal Mismatch}: ``Latest available'' data represents different years across countries
\item \textbf{Causation vs. Correlation}: Statistical correlations shown don't prove causation
\end{enumerate}

\section{Background on the Dataset}

\subsection{Dataset Description}

\begin{itemize}
\item \textbf{Name}: World Development Indicators (WDI)
\item \textbf{Source}: World Bank Open Data (\url{https://data.worldbank.org/indicator})
\item \textbf{Time span}: 1960–present (2024)
\item \textbf{Coverage}: $\sim$217 countries and territories
\item \textbf{Indicators}: Over 1,000 indicators grouped by themes: education, health, environment, economy
\end{itemize}

\subsection{Dataset Organization}

The WDI dataset is structured with columns:
\begin{itemize}
\item Country Name and Code
\item Indicator Name and Code
\item Year (1960-2024)
\item Value (indicator measurement)
\end{itemize}

\subsection{Selected Indicators}

For this analysis, we selected 13 key indicators relevant to SDG 1:

\begin{table}[h]
\centering
\caption{Selected World Development Indicators}
\begin{tabular}{@{}ll@{}}
\toprule
\textbf{Category} & \textbf{Indicator} \\
\midrule
Poverty & Poverty headcount at \$2.15/day (\%) \\
Economic & GDP per capita (constant 2015 US\$) \\
Health & Life expectancy at birth (years) \\
Infrastructure & Access to electricity (\%) \\
Education & Primary enrollment (\% gross) \\
Education & Secondary enrollment (\% gross) \\
Demographic & Total population \\
\bottomrule
\end{tabular}
\end{table}

\section{Literature Review}

\subsection{Key Sources and Prior Research}

\paragraph{World Bank Poverty Reports (2020-2024)}
The World Bank's annual Poverty and Shared Prosperity Reports \cite{worldbank2023} provide the authoritative global poverty trends and methodology. They document the reduction from 36\% (1990) to 9\% (2023) and highlight COVID-19's impact as the first increase in extreme poverty in a generation. However, these reports provide limited focus on Philippines-specific challenges within the SEA context.

\paragraph{UN SDG Progress Reports (2015-2024)}
Official SDG 1 tracking reports \cite{un2015sdg} provide country-level progress assessments and identify success stories (Vietnam, China) and challenges (Sub-Saharan Africa). However, they aggregate regional data and don't deeply compare individual SEA countries.

\paragraph{Academic Research on Southeast Asian Poverty}
Vietnam's dramatic poverty reduction from 50\% (1992) to <2\% (2018) has been extensively studied \cite{worldbank2020vietnam}, identifying key factors including agricultural reforms, rural infrastructure, and education investment. The Asian Development Bank's 2019 report \cite{adb2019} examines why the Philippines' poverty remains high despite economic growth, identifying structural issues: inequality, weak job creation, and limited rural development.

\subsection{Gaps in Existing Literature}

While existing works provide an excellent foundation, this project addresses several gaps:
\begin{enumerate}
\item \textbf{Lack of Three-Tier Comparative Framework}: No study systematically compares Global → SEA → Philippines poverty patterns
\item \textbf{Limited SEA Country Comparisons}: Most studies compare regions, not individual SEA countries
\item \textbf{Static vs. Dynamic Analysis}: We analyze 33-year trajectories with temporal visualizations
\item \textbf{Limited Correlation Analysis in SEA Context}: Our multi-dimensional correlation analysis fills this gap
\end{enumerate}

\section{Methodology}

\subsection{Data Set}

We use 13 World Development Indicators chosen for their:
\begin{itemize}
\item \textbf{Relevance} to SDG 1 and poverty reduction
\item \textbf{Data availability} across countries and years
\item \textbf{Theoretical grounding} in development economics
\end{itemize}

Each indicator represents a dimension of development affecting poverty: education creates opportunities, health enables productivity, infrastructure supports economic activity, and economic growth provides resources for poverty alleviation.

\subsection{Data Preparation}

Data preparation involved several key steps:

\begin{enumerate}
\item \textbf{Data Import}: Loading WDI data from CSV format
\item \textbf{Filtering}: Selecting relevant indicators and countries
\item \textbf{Handling Missing Values}: Identifying and documenting data gaps
\item \textbf{Temporal Alignment}: Using ``latest available year'' approach due to data inconsistency
\item \textbf{Regional Categorization}: Classifying countries into ASEAN, developed, and developing groups
\end{enumerate}

\subsection{Exploratory Data Analysis (EDA)}

We conducted preliminary analysis with three visualization types:

\begin{enumerate}
\item \textbf{Distribution Analysis}: Four histograms showing global, ASEAN, and Philippines distributions for poverty, GDP, life expectancy, and electricity access
\item \textbf{Temporal Heatmap}: 15 countries (ASEAN prioritized) showing poverty evolution over time
\item \textbf{Regional Comparison}: Boxplots comparing poverty distributions across Global, ASEAN, and Philippines
\end{enumerate}

These EDA visualizations revealed:
\begin{itemize}
\item High variance in global poverty (0\% to 85\%+)
\item Philippines positioned near ASEAN median but above global median
\item Strong negative relationship between infrastructure access and poverty
\end{itemize}

\subsection{Data Visualization}

We created 5 polished, publication-quality visualizations:

\begin{enumerate}
\item \textbf{Interactive Choropleth Map} (Figure \ref{fig:choropleth}): Global poverty distribution with 169 countries
\item \textbf{Time Series} (Figure \ref{fig:trends}): ASEAN poverty trends with regional average
\item \textbf{Correlation Heatmap} (Figure \ref{fig:correlation}): Poverty vs. development indicators
\item \textbf{Radar Chart} (Figure \ref{fig:radar}): Philippines vs. ASEAN multi-dimensional comparison
\item \textbf{Interactive Dashboard} (Figure \ref{fig:dashboard}): Philippines deep-dive with 4 indicators
\end{enumerate}

\section{Data Analysis}

\subsection{Global Poverty Landscape}

Based on Visualization 1 (Choropleth Map) and actual data analysis:

\paragraph{Data Coverage}
\begin{itemize}
\item \textbf{169 countries} have poverty headcount data available
\item \textbf{48 countries} lack poverty measurements
\item \textbf{Latest data year}: Varies by country (2018-2024)
\end{itemize}

\paragraph{Global Statistics (Latest Available Year per Country)}
\begin{itemize}
\item \textbf{Global median poverty}: 2.70\%
\item \textbf{Global mean poverty}: 14.03\%
\item \textbf{Wide variation}: From 0\% (several developed nations) to 85\%+ (some sub-Saharan African countries)
\end{itemize}

\paragraph{Geographic Patterns}
Sub-Saharan Africa shows the highest poverty concentrations, while East Asia demonstrates dramatic success stories (China, Vietnam, Indonesia). Latin America shows moderate to low poverty, and Europe \& North America achieve near-zero extreme poverty.

\paragraph{Historical Trend}
\begin{itemize}
\item \textbf{1981 average}: 17.22\% global poverty
\item \textbf{2024 average}: 4.23\% global poverty
\item \textbf{Total reduction}: 12.99 percentage points over 43 years
\end{itemize}

\subsection{ASEAN Regional Analysis}

Only 7 out of 11 ASEAN countries have poverty data available:

\begin{table}[h]
\centering
\caption{ASEAN Poverty Rates (Latest Available Year)}
\begin{tabular}{@{}lr@{}}
\toprule
\textbf{Country} & \textbf{Poverty Rate (\%)} \\
\midrule
Malaysia & 0.0 \\
Thailand & 0.0 \\
Philippines & 5.3 \\
Indonesia & 5.4 \\
Myanmar & 10.3 \\
Lao PDR & 15.7 \\
Timor-Leste & 43.9 \\
\midrule
ASEAN Median & 5.40 \\
ASEAN Mean & 11.51 \\
\bottomrule
\end{tabular}
\end{table}

\textbf{Key Findings}:
\begin{itemize}
\item Malaysia and Thailand achieved zero extreme poverty
\item Philippines (5.3\%) slightly below ASEAN median (5.4\%)
\item Timor-Leste faces severe poverty (43.9\%), significantly above regional average
\item ASEAN median (5.4\%) is higher than global median (2.7\%)
\end{itemize}

\subsection{Development Indicators and Poverty Correlation}

Based on actual calculations from our complete dataset, we found strong correlations:

\begin{table}[h]
\centering
\caption{Correlation Coefficients with Poverty Headcount}
\begin{tabular}{@{}lr@{}}
\toprule
\textbf{Indicator} & \textbf{Correlation (r)} \\
\midrule
Electricity access & -0.847*** \\
Life expectancy & -0.786*** \\
GDP per capita & -0.397** \\
\bottomrule
\multicolumn{2}{l}{\footnotesize ***p < 0.001, **p < 0.01}
\end{tabular}
\end{table}

\paragraph{Interpretation}
\begin{enumerate}
\item \textbf{Infrastructure is the strongest predictor}: Electricity access correlation of -0.847 suggests infrastructure access is critical for poverty reduction
\item \textbf{Health outcomes matter significantly}: Life expectancy shows strong correlation (-0.786), indicating better health enables economic participation
\item \textbf{Economic growth shows moderate correlation}: GDP per capita at -0.397 is weaker than expected, suggesting income growth alone is insufficient—distribution and access to services matter more than raw income
\end{enumerate}

\subsection{Philippines-Specific Insights}

\paragraph{Current Status (2023)}
\begin{itemize}
\item \textbf{Poverty rate}: 5.3\%
\item \textbf{Position}: Slightly below ASEAN median (5.4\%), but above global median (2.7\%)
\end{itemize}

\paragraph{Multi-Dimensional Comparison}
The radar chart visualization reveals Philippines shows comparable performance across most development indicators, with strengths in infrastructure access (electricity) and balanced development profile—neither leader nor laggard in the ASEAN region.

\subsection{Country-Level Progress Analysis}

\paragraph{Progress Categorization (120 countries with multi-year data)}
\begin{itemize}
\item \textbf{Improving} (>5\% reduction): 65 countries (54.2\%)
\item \textbf{Stable} (±5\%): 49 countries (40.8\%)
\item \textbf{Worsening} (>5\% increase): 6 countries (5.0\%)
\end{itemize}

\paragraph{Top 5 Poverty Reduction Success Stories}
\begin{enumerate}
\item China: -97.0\% reduction
\item Indonesia: -80.8\% reduction
\item Nepal: -80.3\% reduction
\item Uzbekistan: -77.1\% reduction
\item Guinea: -74.7\% reduction
\end{enumerate}

Indonesia's remarkable success (-80.8\%) provides a relevant comparison for the Philippines given similar regional context and development challenges.

\section{Results and Discussion}

\subsection{Visualization Results}

Our five main visualizations provide comprehensive insights into global poverty patterns:

\begin{figure}[h]
\centering
\includegraphics[width=0.45\textwidth]{../figures/viz_01_choropleth.html}
\caption{Global Poverty Headcount Distribution Map (Interactive). Dark red indicates higher poverty rates, with Sub-Saharan Africa showing highest concentrations.}
\label{fig:choropleth}
\end{figure}

\begin{figure}[h]
\centering
\includegraphics[width=0.45\textwidth]{../figures/viz_02_trends.png}
\caption{ASEAN Poverty Trends Over Time. Philippines (red) and ASEAN average (black dashed) show declining trends, though at different rates.}
\label{fig:trends}
\end{figure}

\begin{figure}[h]
\centering
\includegraphics[width=0.45\textwidth]{../figures/viz_03_correlation.png}
\caption{Correlation Matrix: Poverty vs Development Indicators. Strong negative correlation with electricity access (-0.847) and life expectancy (-0.786).}
\label{fig:correlation}
\end{figure}

\begin{figure}[h]
\centering
\includegraphics[width=0.45\textwidth]{../figures/viz_04_radar.png}
\caption{Multi-Dimensional Comparison: Philippines vs ASEAN Average. All values normalized to 0-1 scale showing balanced development profile.}
\label{fig:radar}
\end{figure}

\begin{figure}[h]
\centering
\includegraphics[width=0.45\textwidth]{../figures/viz_05_philippines_dashboard.html}
\caption{Philippines Development Dashboard (Interactive). Four-panel view showing poverty, GDP, education, and infrastructure trends over time.}
\label{fig:dashboard}
\end{figure}

\subsection{Key Findings Summary}

\begin{enumerate}
\item \textbf{Global Progress}: Extreme poverty reduced from 17.22\% (1981) to 4.23\% (2024)—a remarkable achievement

\item \textbf{Infrastructure Critical}: Electricity access shows the strongest correlation with poverty reduction (r = -0.847), stronger than GDP per capita

\item \textbf{Philippines Position}: At 5.3\% poverty, Philippines performs slightly better than ASEAN median but lags global median

\item \textbf{ASEAN Diversity}: Wide variation from 0\% (Malaysia, Thailand) to 43.9\% (Timor-Leste)

\item \textbf{Majority Improving}: 54.2\% of countries with data show >5\% poverty reduction

\item \textbf{Health-Poverty Link}: Life expectancy correlation of -0.786 emphasizes health system importance
\end{enumerate}

\section{Conclusion}

\subsection{Summary of Key Insights}

This analysis demonstrates that \textbf{poverty reduction is achievable but requires sustained, multi-dimensional effort}. The global reduction from 17.22\% to 4.23\% over 43 years represents over a billion people lifted from extreme poverty—a historic achievement.

However, the data reveals critical insights often overlooked:

\begin{enumerate}
\item \textbf{Infrastructure matters more than income}: Electricity access (r = -0.847) correlates more strongly than GDP per capita (r = -0.397)
\item \textbf{Health is wealth}: Life expectancy correlation (r = -0.786) emphasizes human capital importance
\item \textbf{Success is possible}: Malaysia, Thailand, China, Indonesia prove dramatic poverty reduction achievable in developing Asian contexts
\item \textbf{Philippines has potential}: At 5.3\%, Philippines outperforms ASEAN median but has clear path to Malaysia/Thailand levels (0\%)
\end{enumerate}

\subsection{Relationship to SDG 1}

Our findings demonstrate significant global progress toward SDG 1 targets:

\begin{itemize}
\item \textbf{SDG 1.1} (Eradicate extreme poverty by 2030): Global average reduced from 17.22\% to 4.23\%—on trajectory but not yet achieved
\item \textbf{SDG 1.2} (Reduce poverty by 50\% per national definitions): Many countries exceeded this (China -97\%, Indonesia -80.8\%)
\item \textbf{Multi-dimensional poverty}: Evidence supports integrated approach combining infrastructure, health, and economic growth
\end{itemize}

\subsection{Policy Recommendations}

\paragraph{For Philippines (current 5.3\% poverty)}
\begin{enumerate}
\item \textbf{Infrastructure acceleration}: Given strongest correlation (r = -0.847), continue electricity expansion
\item \textbf{Learn from regional success}: Study Malaysia and Thailand's path to 0\% poverty
\item \textbf{Health systems investment}: Life expectancy correlation (r = -0.786) suggests health outcomes critical
\item \textbf{Move beyond income growth}: Focus on access and distribution, not just GDP growth
\end{enumerate}

\paragraph{For ASEAN Region}
\begin{enumerate}
\item \textbf{Address inequality}: ASEAN median (5.4\%) above global (2.7\%) despite economic growth
\item \textbf{Support Timor-Leste}: At 43.9\% poverty, needs coordinated regional assistance
\item \textbf{Share best practices}: Malaysia and Thailand models should inform other countries
\item \textbf{Data improvement}: 4 countries lack poverty data—measurement essential for tracking
\end{enumerate}

\subsection{Limitations and Future Work}

\paragraph{Data Limitations}
\begin{itemize}
\item Missing poverty data for 48 of 217 countries (22\%)
\item ASEAN gaps: Only 7 of 11 countries have data
\item Temporal inconsistency: ``Latest year'' varies 2018-2024
\item Country-level aggregates mask within-country disparities
\end{itemize}

\paragraph{Future Research Directions}
\begin{enumerate}
\item \textbf{Causal analysis}: Use panel regression to establish causal relationships
\item \textbf{Within-country analysis}: Examine sub-national poverty data
\item \textbf{Time-series forecasting}: Project poverty trajectories to 2030
\item \textbf{Multi-dimensional poverty}: Integrate UNDP Multi-dimensional Poverty Index
\item \textbf{Machine learning prediction}: Build models to identify at-risk countries
\end{enumerate}

\subsection{Final Reflection}

The question is not whether poverty can be eliminated—the data shows it can—but whether we will summon the political will and resource allocation to complete the journey. The Philippines, positioned slightly below ASEAN median with balanced development indicators, has both the foundation and regional examples to make the final push toward SDG 1 targets.

As the 2030 SDG deadline approaches, the data provides both hope and urgency: remarkable progress achieved, but persistent effort required to reach zero poverty.

%%
%% The acknowledgments section is defined using the "acks" environment
%% (and NOT an unnumbered section). This ensures the proper
%% identification of the section in the article metadata, and the
%% consistent spelling of the heading.
\begin{acks}
We thank the World Bank Open Data initiative for providing comprehensive development indicators, the United Nations Development Programme for SDG tracking frameworks, and our course instructors for guidance throughout this project.
\end{acks}

%%
%% The next two lines define the bibliography style to be used, and
%% the bibliography file.
\bibliographystyle{ACM-Reference-Format}
\begin{thebibliography}{9}

\bibitem{worldbank2023}
World Bank. 2023.
\newblock \emph{Poverty and Shared Prosperity Report}.
\newblock World Bank, Washington, DC.

\bibitem{un2015sdg}
United Nations. 2015.
\newblock \emph{Sustainable Development Goals}.
\newblock UN General Assembly.

\bibitem{roser2023}
Roser, M., and Ortiz-Ospina, E. 2023.
\newblock Global Extreme Poverty.
\newblock \emph{Our World in Data}.

\bibitem{worldbank2020vietnam}
World Bank. 2020.
\newblock \emph{Vietnam's Success in Poverty Reduction}.
\newblock World Bank, Washington, DC.

\bibitem{adb2019}
Asian Development Bank. 2019.
\newblock \emph{The Philippines Poverty Puzzle}.
\newblock ADB, Manila.

\bibitem{unesco2021}
UNESCO. 2021.
\newblock \emph{Education and Poverty: Systematic Review}.
\newblock UNESCO, Paris.

\bibitem{adb2018infra}
Asian Development Bank. 2018.
\newblock \emph{Infrastructure and Poverty Reduction in Asia}.
\newblock ADB, Manila.

\bibitem{ravallion2019}
Ravallion, M., and Chen, S. 2019.
\newblock Global Poverty Measurement when Relative Income Matters.
\newblock \emph{Journal of Public Economics} 177, 104046.

\end{thebibliography}

\end{document}
